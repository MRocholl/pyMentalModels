\documentclass{scrartcl}
\usepackage{german}
\usepackage[german]{babel}

% zusätzliche mathematische Symbole, AMS=American Mathematical Society 
\usepackage{amssymb}
\usepackage{amsmath}
\usepackage{wasysym}
\usepackage{ dsfont }
\usepackage{multicol}
\setlength{\columnsep}{5mm}

% fürs Einbinden von Graphiken
\usepackage{graphicx}
\usepackage{float}
\usepackage{tikz}
\usetikzlibrary{shapes,arrows}
\usepackage{syntax}
\usepackage{pgfplots}

% für Namen etc. in Kopf- oder Fußzeile
\usepackage{fancyhdr}


% erlaubt benutzerdefinierte Kopfzeilen 
\pagestyle{fancy}

% Definition der Kopfzeile
\lhead{
\begin{tabular}{ll}
Moritz Rocholl\\
\end{tabular}
}
\chead{Modal Mental Model reasoner}
\rhead{}
\lfoot{}
\cfoot{Seite \thepage}
\rfoot{} 
\begin{document}

\section*{Mental Model Reasoner}

\subsection*{Parsing Grammar}

The Parsing grammar is specified in the Bachus Naur Form:
\begin{grammar}
    <atom> ::= <string_without_spaces_and_tabs>\\
    <necessary> ::= "[]"\\
    <possibly> ::= "<>"\\
    <not> ::= "~"\\
    <and> ::= "&"\\
    <or> ::= "|"\\
    <xor> ::= "^"\\
    <imply>::= "->"\\
    <biconditional>::= "<->"\\
    <operation> ::=  <atom> | <necessary> <operation> | <possibly>
    <operation> | <not> <operation> |
		   <operation><and><operation> | <operation><or><operation>
                   |
		   <operation> <xor> <operation> | <operation> <implies>
                   <operation> | <operation> <biconditional> <operation>\\
    <expr> ::= <operation> | <operation> <expr>
\end{grammar}

 Process modell of the modal logic reasoning implementation:

\begin{enumerate}
    \item [Step 1] Expression is parsed by the parsing algorithm based on the
        grammar introduced above.\\
    \item [Step 2] Convert the list representing the parsed expression into a
        string that can be interpreted by the sympy library.\\
    \item [Step 3] The function sympify from the sympy-library then processes
        the expression in a manner defined by the normal logical rules. The
        rules are here redefined to better represent the "human" understanding
        of logical laws. e.g. xor instead of or.\\
    \item [Step 4] The sympified expression is an object of a the logical class
        that has the least depth.
        Using the attributes of the logical object, the atoms are cached within
        it as an attribute, we hence populate an array of the dimension $|Atoms|$.
        I.e. Xor(Bread, Butter, Milk) $\to 3 D$.
        Each Atom has one column and each row is one possibility or "possible world".
        For each logical operator, different rules apply to the process of populating the model.
        Step 4 is repeated recursively for each operator in the expression.
    \item [Step5] The final matrix is then evaluated using bitmaps of the different rows and columns.
        Possible inferences are returned.
\end{enumerate}


The parsed expression:
\begin{grammar}
    <Expr> ::= "Bread ^ Butter ^ Salad"
\end{grammar}   
yields the array:
\begin{center}
    ['Bread', '\^{}', 'Butter', '\^{}' 'Salad']
\end{center}
After formatting: 
\begin{center}
Xor(Bread, Butter, Salad)
\end{center}
After passing the string to the sympify function:
\begin{center}
 Xor(Bread, Butter, Salad)\\
Object-type: Xor\\
    Xor \to attributes := \{Bread, Butter, Salad\}\\
\end{center}

Repeat evaluation with the original expression substituting Xor by Or.

\subsection*{Flow chart representation of algorithm}
\label{sub:flow_chart_representation_of_algorithm}


% Define block styles
\tikzstyle{decision} = [diamond, draw, fill=blue!20, 
    text width=4.5em, text badly centered, node distance=3cm, inner sep=0pt]
\tikzstyle{block} = [rectangle, draw, fill=blue!20, 
    text width=5em, text centered, rounded corners, minimum height=4em]
\tikzstyle{line} = [draw, -latex']
\tikzstyle{cloud} = [ellipse, draw,fill=red!20, node distance=3cm,
    minimum height=3em, text width=4.5em, text badly centered]
    

\begin{tikzpicture}[node distance = 2cm, auto]
    % Place nodes
    \node [block] (init) {input  expression};
    \node [cloud, right of=init] (system) {intuitive or full model};
    \node [block, below of=init] (parse) {Parse expression};
    \node [block, below of=parse] (format) {Format parsed expression};
    \node [block, below of=format] (evaluate) {Simplify expression};
    \node [block, below of=evaluate] (initialize) {Initialize or update array};
    \node [cloud, below of=system, node distance=2.5cm] (operators) {Definition of logical operators};
    \node [block, left of=evaluate, node distance=3cm] (update) {recursively evaluate next};
    \node [decision, below of=initialize] (decide) {maximum formula depth?};
    \node [block, below of=decide, node distance=3cm] (inference) {Make modal logical inference};
    % Draw edges
    \path [line] (init) -- (parse);
    \path [line] (parse) -- (format);
    \path [line] (format) -- (evaluate);
    \path [line] (evaluate) -- (initialize);
    \path [line] (initialize) -- (decide);
    \path [line] (decide) -| node [near start] {no} (update);
    \path [line] (update) |- (evaluate);
    \path [line] (decide) -- node {yes}(inference);
    \path [line,dashed] (system) -- (init);
    \path [line,dashed] (system) -- (operators);
    \path [line,dashed] (operators) |- (evaluate);
    \path [line,dashed] (operators) |- (initialize);
\end{tikzpicture}


\end{document}
